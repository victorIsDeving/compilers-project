\documentclass[10pt, a4paper]{article} 
\addtolength{\hoffset}{-2.25cm}
\addtolength{\textwidth}{4.5cm}
\addtolength{\voffset}{-3.25cm}
\addtolength{\textheight}{5cm}
\setlength{\parskip}{0pt}
\setlength{\parindent}{0in}

%----------------------------------------------------------------------------------------
%	PACKAGES AND OTHER DOCUMENT CONFIGURATIONS
%----------------------------------------------------------------------------------------

\usepackage{blindtext} % Package to generate dummy text
\usepackage{charter} % Use the Charter font
\usepackage[utf8]{inputenc} % Use UTF-8 encoding
\usepackage{microtype} % Slightly tweak font spacing for aesthetics
\usepackage[english, portuguese]{babel} % Language hyphenation and typographical rules
\usepackage{amsthm, amsmath, amssymb} % Mathematical typesetting
\usepackage{float} % Improved interface for floating objects
\usepackage[final, colorlinks = true, 
            linkcolor = black, 
            citecolor = black]{hyperref} % For hyperlinks in the PDF
\usepackage{graphicx, multicol} % Enhanced support for graphics
\usepackage{xcolor} % Driver-independent color extensions
\usepackage{marvosym, wasysym} % More symbols
\usepackage{rotating} % Rotation tools
\usepackage{censor} % Facilities for controlling restricted text
\usepackage{listings, style/lstlisting} % Environment for non-formatted code, !uses style file!
\usepackage{pseudocode} % Environment for specifying algorithms in a natural way
\usepackage{style/avm} % Environment for f-structures, !uses style file!
\usepackage{booktabs} % Enhances quality of tables
\usepackage{tikz-qtree} % Easy tree drawing tool
\tikzset{every tree node/.style={align=center,anchor=north},
         level distance=2cm} % Configuration for q-trees
\usepackage{style/btree} % Configuration for b-trees and b+-trees, !uses style file!
\usepackage[
    backend=biber,
    style=numeric,
    sorting=nyt]{biblatex} % Complete reimplementation of bibliographic facilities
\addbibresource{references.bib}
\usepackage{csquotes} % Context sensitive quotation facilities
\usepackage[yyyymmdd]{datetime} % Uses YEAR-MONTH-DAY format for dates
\renewcommand{\dateseparator}{-} % Sets dateseparator to '-'
\usepackage{fancyhdr} % Headers and footers
\pagestyle{fancy} % All pages have headers and footers
\fancyhead{}\renewcommand{\headrulewidth}{0pt} % Blank out the default header
\fancyfoot[L]{} % Custom footer text
\fancyfoot[C]{} % Custom footer text
\fancyfoot[R]{\thepage} % Custom footer text
\newcommand{\note}[1]{\marginpar{\scriptsize \textcolor{red}{#1}}} % Enables comments in red on margin

%----------------------------------------------------------------------------------------

\graphicspath{ {./resources/} }

\begin{document}

\fancyhead[C]{}
\hrule \medskip 
\begin{minipage}{0.295\textwidth} 
    \raggedright
    \footnotesize
    Victor Augusto Costa Monteiro \hfill\\   
    % NºUSP 8942937 \hfill\\
    victor.augusto.monteiro@usp.br
\end{minipage}
\begin{minipage}{0.4\textwidth} 
    \centering 
    \large 
    Proposta de Projeto Final\\ 
    \normalsize 
    Linguagens e Compiladores\\CPGEI IN0018 e PPGCA CAES101\\ 
\end{minipage}
\begin{minipage}{0.295\textwidth} 
    \raggedleft
    \today\hfill\\
\end{minipage}
\medskip\hrule 

\section{Descrição do Projeto}

O objetivo deste projeto é desenvolver um compilador completo para uma linguagem 
de programação chamada \textbf{Cimples} (C Simplified), que segue a sintaxe e 
semântica básica da linguagem C, porém com recursos simplificados.

O compilador será implementado usando Flex para análise léxica, Bison para 
análise sintática, e irá gerar código intermediário em formato de 
três endereços. O projeto incluirá todas as fases de compilação exigidas: 
análise léxica, sintática, semântica e geração de código.

A princípio o trabalho será desenvolvido para o Conceito C de avaliação. 
Após o término de um conceito eu vou passar para o seguinte. 
Irei conceito a conceito conforme minha disponibilidade de tempo 
para trabalharno projeto, dado que já trabalho e estou cursando 
algumas outras matérias também.


\section{Linguagem Fonte - Cimples}

A linguagem Cimples terá a seguinte gramática BNF:

\begin{verbatim}
    <program> ::= <declaration-list>

    <declaration-list> ::= <declaration> | <declaration-list> <declaration>

    <declaration> ::= <var-declaration> 

    <var-declaration> ::= <type-specifier> IDENTIFIER ";" 
    | <type-specifier> IDENTIFIER "[" NUMBER "]" ";"

    <type-specifier> ::= "int" | "float" | "char"

    <compound-stmt> ::= "{" <local-declarations> <statement-list> "}"

    <local-declarations> ::= <empty> | <local-declarations> <var-declaration>

    <statement-list> ::= <empty> | <statement-list> <statement>

    <statement> ::= <expression-stmt> 
    | <selection-stmt> 
    | <compound-stmt> 
    | <return-stmt> 
    | <iteration-stmt> 
    | <io-stmt>

    <expression-stmt> ::= <expression> ";" | ";"

    <selection-stmt> ::= "if" "(" <expression> ")" <statement> 
    | "if" "(" <expression> ")" <statement> "else" <statement>

    <iteration-stmt> ::= "while" "(" <expression> ")" <statement>
    | "for" "(" <expression> ";" <expression> ";" <expression> ")" <statement>

    <return-stmt> ::= "return" ";" | "return" <expression> ";"

    <io-stmt> ::= "read" "(" IDENTIFIER ")" ";" | "write" "(" <expression> ")" ";"

    <expression> ::= <var> "=" <expression> | <simple-expression>

    <var> ::= IDENTIFIER | IDENTIFIER "[" <expression> "]"

    <simple-expression> ::= <additive-expression> 
    | <simple-expression> <relop> <additive-expression>

    <relop> ::= "<=" | "<" | ">" | ">=" | "==" | "!="

    <additive-expression> ::= <term> 
    | <additive-expression> <addop> <term>

    <addop> ::= "+" | "-"

    <term> ::= <factor> | <term> <mulop> <factor>

    <mulop> ::= "*" | "/"

    <factor> ::= "(" <expression> ")" | <var> | NUMBER | STRING | CHAR

\end{verbatim}

\section{Características da Linguagem}

A linguagem Cimples incluirá:

\begin{itemize}
    \item Três tipos de variáveis: \texttt{int}, \texttt{float} e \texttt{char}
    \item Comandos de escrita (\texttt{write}) e leitura (\texttt{read})
    \item Estrutura de decisão: \texttt{if-else}
    \item Estruturas de repetição: \texttt{while} e \texttt{for}
    \item Declaração de funções com parâmetros e valor de retorno
    \item Arrays unidimensionais
    \item Expressões aritméticas e lógicas
\end{itemize}

\section{Linguagem Alvo}

O compilador irá gerar código intermediário em \textbf{três endereços}, 
que pode ser posteriormente traduzido para código de máquina ou outra 
linguagem de alto nível. O formato básico do código de três endereços será:

\begin{verbatim}
    x = y op z
    if x relop y goto L
    L:
    param x
    call f, n
\end{verbatim}

\section{Componentes do Compilador}

O compilador será composto pelos seguintes módulos:

\begin{enumerate}
    \item \textbf{Analisador Léxico}: 
    Implementado em Flex, responsável por reconhecer os tokens da linguagem
    \item \textbf{Analisador Sintático}: 
    Implementado em Bison, responsável por validar a estrutura do programa
    \item \textbf{Analisador Semântico}: 
    Responsável por verificação de tipos e outras regras contextuais
    \item \textbf{Gerador de Código}: 
    Responsável por produzir o código em três endereços
    \item \textbf{Tabela de Símbolos}: 
    Estrutura para armazenar informações sobre identificadores
\end{enumerate}

\section{Previsões de expansão para os conceitos subsequentes}

Para o Conceito B, as principais adições previstas na gramática serão:
\begin{verbatim}
    <var-declaration> ::= <type-specifier> IDENTIFIER ";" 
    | <type-specifier> IDENTIFIER "[" NUMBER "]" "[" NUMBER "]" ";"
    | <type-specifier> IDENTIFIER "[" NUMBER "]" ";" 

    <var> ::= IDENTIFIER 
    | IDENTIFIER "[" <expression> "]" "[" <expression> "]"
    | IDENTIFIER "[" <expression> "]" 

    <selection-stmt> ::= "if" "(" <expression> ")" <statement> 
    | "switch" "(" <expression> ")" "{" <case-list> "}"
    | "if" "(" <expression> ")" <statement> "else" <statement>

    <case-list> ::= <case> <case-list> | <case>
    <case> ::= "case" <constant> ":" <statement-list> 
    | "default" ":" <statement-list>
    
    <iteration-stmt> ::= "while" "(" <expression> ")" <statement>
    | "repeat" <statement-list> "until" "(" <expression> ")" ";"
    | "for" "(" <expression> ";" <expression> ";" <expression> ")" <statement>
\end{verbatim}

Para o Conceito A as adições previstas subsquentes, além das implementações léxicas, serão:
\begin{verbatim}
    <param> ::= <type-specifier> IDENTIFIER 
    | <type-specifier> IDENTIFIER "[" "]" 
    | "ref" <type-specifier> IDENTIFIER 

    <expression> ::= ...
    | "alloc2D" "(" <type-specifier> "," <expression> "," <expression> ")" 
    | "alloc" "(" <type-specifier> "," <expression> ")" 
\end{verbatim}

\end{document}